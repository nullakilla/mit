\documentclass[a4paper,14pt,ukrainian]{extarticle}

\usepackage[T2A]{fontenc}
\usepackage[utf8]{inputenc}
\usepackage[english,ukrainian]{babel}


\usepackage[dvips]{graphicx}
\graphicspath{{images/}}

\usepackage{geometry}
\geometry{left=2.5cm}
\geometry{right=1.5cm}
\geometry{top=2.5cm}
\geometry{bottom=2.5cm}

\usepackage{indentfirst}

\usepackage{enumitem}

\sloppy

\begin{document}
\setlength{\parindent}{1.27cm}
\setlength{\parskip}{0pt}
\pagestyle{empty}

\setlist{nolistsep}



\begin{titlepage}
\newpage

\begin{center}
\normalfont\scshape{Міністерство освіти і науки, молоді та спорту України} \\
\normalfont\scshape{Чернігівський державний технологічний університет} \\
\normalfont\scshape{Кафедра інформаційних та комп'ютерних систем} \\
\end{center}

\vspace{2em}

\begin{flushright}
{\bf \MakeUppercase{затверджую}}
\end{flushright}

\vspace{1em}

\begin{flushright}
Проректор з науково-\\
педагогічної роботи \\
\vspace{1em}
Кальченко В.В. \\
\vspace{1em}
"...." ............... 2011 р.
\end{flushright}

\vspace{4em}

\begin{center}
{\large \MakeUppercase{\bf {\textsf{навчальна програма}}}} \\
з нормативної дисципліни \\
{\bf "Мережеві інформаційні технології"} \\
підготовки фахівців освітньо-кваліфікаційного рівня "спеціаліст", "магістр" \\
для спеціальності 7.091501, 8.091501 "Комп’ютерні системи та мережі",
7.091502, 8.011502 "Системне програмування"
\end{center}

\vspace{4em}

\begin{flushright}
Обговорено і рекомендовано \\
на засіданні кафедри \\
Протокол №... \\
від "...." ............... 2011 р. \\
завідувач кафедри \\
д.т.н., проф. В.В. Казимир \\
\rule{5cm}{0.5pt} \\
\end{flushright}

\vspace{\fill}

\begin{center}
Чернігів --- 2012
\end{center}

\end{titlepage}



\newcommand{\ssection}[1]{%
  {\normalsize \newpage\section[#1]{\centering\textbf{\scshape #1}}}}
\newcommand{\ssubsection}[1]{%
  \subsection[#1]{\centering\textbf{#1}}}



  
\ssection{мета та завдання дисціпліни}

Дисципліна  "Мережеві інформаційні технології" входить до переліку нормативних дисциплін професійної та практичної підготовки освітньо-професійної програми вищої освіти за  напрямком підготовки 0915 "Комп'ютерна інженерія" за  спеціальністями 7.091501, 8.091501 "Комп'ютерні системи та мережі" та 7.091502, 8.091502 "Системне програмування" і розглядається катедрою як завершальна частина підготовки спеціаліста та магістра. Необхідною передумовою вивчення дисципліни "Мережеві інформаційні технології" є засвоєння матеріалу дисциплін "Організація БД" та "Комп'ютерні мережі", які передують їй в навчальному процесі.

Метою викладання дисципліни "Мережеві інформаційні технології" є необхідність формування у студентів чіткої системи уявлень про цілісний комплекс проблем, що мають бути вирішені  у процесі проектування та розробки розподілених  інформаційних систем.

Після вивчення дисципліни студенти \textbf{повинні знати}:
\begin{itemize}
\setlength{\parskip}{0pt}
\setlength{\parsep}{0pt}
\setlength{\itemsep}{1pt}
\item засоби інтеграції інформаційних систем;
\item існуючі засоби забезпечення надійності функціонування інформаційних систем;
\item засоби захисту інформаційних систем від несанкціонованого доступу;
\item можливості сучасних серверів БД;
\item інструментальні засоби моніторингу комп'ютерної мережі;
\item типи ліцензій на ПЗ, що використовується в дисципліні.
\end{itemize}

У результаті опанування навчальною дисципліною студенти 
\textbf{повинні вміти}:
\begin{itemize}
\setlength{\parskip}{0pt}
\setlength{\parsep}{0pt}
\setlength{\itemsep}{1pt}
\item проводити інтеграцію локальних обчислювальних мереж у єдину інформаційну систему;
\item проектувати розподілені бази даних;
\item обирати мережеву ОС, сервер БД та засоби для моніторингу мережі;
\item імплементовувати  інформаційні системи з архітектурою "клієнт-сервер";
\item підвищувати надійність збереження даних у БД;
\item оптимізовувати взаємодію інформаційної системи з БД за рахунок використання тригерів та процедур, що зберігаються;
\item виконувати моніторинг стану комп'ютерної мережі;
\end{itemize}

\textbf{Значення} дисципліни полягає в тому, що вона дає можливість студентам оволодіти сучасними методами проектування ІС та навчитися розробляти їх з використанням архітектури "клієнт-сервер". Крім цього, студенти вивчають можливості підвищення надійності збереження даних на сучасних серверах баз даних та засоби моніторингу програм та хостів мережі.

\ssection{зміст дисципліни}

\ssubsection{Вступ}
Предмет та завдання курсу “Мережеві інформаційні технології. Принципи проектування сучасних інформаційних систем (IC). Моделі взаємодії у архітектурі "клієнт-сервер". Моніторинг корпоративних IC та мереж.

\ssubsection{Забезпечення надійності роботи баз даних}
Підвищення надійності роботи обчислювальних мереж. Надійність систем з БД. Пам'ять з корекцією помилок. Сховища даних та RAID-масиви. Пристрої безперервного живлення. Кластерізація серверних систем.

\ssubsection{Реплікація баз даних}
Поняття реплікації. Види реплікації. Встановлення сучасного сервера БД MySQL.
Встановлення сучасного сервера БД PostgreSQL. Налаштування реплікації БД у сервері MySQL. Налаштування реплікації БД у сервері PostgreSQL. Особливості роботи серверів БД при виконанні реплікації. Вплив реплікації на надійність збереження даних.

\ssubsection{Тригери та процедури, що зберігаються}
Тригери. Тригери у MySQL. Тригери у PostgreSQL. Тригерні функції. Процедури, що зберігаються. Вплив тригерів та процедур, що зберігаються на надійність збереження даних та швидкодію сервера БД.

\ssubsection{Простий протокол управління мережею SNMP}
Опис протокола SNMP. Агенти та керуючі системи. Методи протокола SNMP. Принцип роботи протокола SNMP. Поняття MIB. Встановлення вільної реалізації сервера SNMP. Встановлення вільних програм для роботи з SNMP. Програма snmpwalk. Перехоплення пакетів SNMP за допомогою програми-сніффера. Протоколи, зверху яких працює SNMP.

\ssubsection{Моніторинг мережі за допомогою MRTG}
Призначення MRTG. Встановлення MRTG в ОС. Початкова конфігурація MRTG. Встановлення та налаштування HTTP-сервера для можливості отримання даних моніторингу по протоколу HTTP. Конфігурація MRTG для ґенерації користувацьких графіків. Конфігурація ОС для періодичного запуску MRTG за допомогою демона cron.

\ssubsection{Моніторинг мережі за допомогою Nagios}
Призначення Nagios. Ріні моделі мережі, на яких працює Nagios. Встановлення Nagios в ОС. Початкова конфігурація Nagios. Налаштування HTTP-сервера для підключення веб-інтерфейсу Nagios. Шляхи сповіщення системного адміністратора про несправності мережі. Моніторинг машин зі встановленими різними ОС.

\ssubsection{Моніторинг виконання Java програм}
JMX. Програма JConsole. Моніторинг Java-процесів. Моніторинг пам'яті. Типи пам'яті у Java Virtual Machine. Моніторинг завантажених у віртуальну машину класів. Виклик збиральника сміття.

\ssection{перелік навчально-методичної літератури}

\ssubsection{Навчальна література}

\begin{enumerate}
\item Васильев А.Ю. Работа с Postgresql: настройка, масштабирование. Справочное пособие, 2011. --- 152 с. [Електронний ресурс]. --- Режим доступу до книги: http://postgresql.leopard.in.ua/.
\item MRTG --- Tobi Oetiker’s MRTG --- The Multi Router Traffik Grapher [Електронний ресурс]. --- Режим   доступу: URL: http://oss.oetiker.ch/mrtg/. --- Назва з екрану.
\item Nagios --- The Industry Standard in IT Infrastructure Monitoring [Електронний ресурс]. --- Режим доступу: URL: http://www.nagios.org/. --- Назва з екрану.
\item PostgreSQL: The world’s most advanced open source database [Електронний ресурс]. --- Режим доступу: URL: http://www.postgresql.org/. --- Назва з екрану.
\end{enumerate}

\ssubsection{Методична література}

\begin{enumerate}
\item Мережеві інформаційні технології. Методичні вказівки до виконання лабораторних робіт для студентів спеціальностей "Комп'ютерні системи та мережі", "Системне програмування" Укл.: Хижняк А.В., Пріла О.А. --- Чернігів: ЧДТУ, 2011. --- 48 с., рос. мовою
\end{enumerate}

\end{document}
